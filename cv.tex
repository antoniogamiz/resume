% !TEX TS-program = xelatex
% !TEX encoding = UTF-8

%%%%%%%%%%%%%%%%%%%%%%%%%%%%%%%%%%%%%%%%%%%%%%%%%%%%%%%%%%%%%%%%%
%% Resume template - https://github.com/lquesada/resume
%% Check that URL for licensing information.
%% 
%% Copyright (c) 2019, Luis Quesada Torres -
%% https://github.com/lquesada | www.luisquesada.com
%%
%% Heavily based on (and old version of) the SIMPLE-RESUME-CV
%% by zachscrivena, which is unlicensed.
%% <https://github.com/zachscrivena/simple-resume-cv>
%%%%%%%%%%%%%%%%%%%%%%%%%%%%%%%%%%%%%%%%%%%%%%%%%%%%%%%%%%%%%%%%%

%%%%%%%%%%%%%%%%%%%%%%%%%%%%%%%%%%%%%%%%%%%%%%%%%%%%%%%%%%%%%%%%%
%% INSTRUCTIONS FOR COMPILING THIS DOCUMENT ("resume.tex")
%% TEX ---(XeLaTeX)---> PDF:
%%
%% Method 1: Use latexmk for fully automated document generation:
%%   latexmk -xelatex "CV.tex"
%%   (add the -pvc switch to automatically recompile on changes)
%%
%% Method 2: Use XeLaTeX directly:
%%   xelatex "CV.tex"
%%   (run multiple times to resolve cross-references if needed)
%%
%% PDF ---(pdf2htmlEX)> HTML:
%%
%% Install pdf2htmlEX from https://github.com/coolwanglu/pdf2htmlEX, then:
%%   alias pdf2htmlEX='docker run -ti --rm -v `pwd`:/pdf bwits/pdf2htmlex pdf2htmlEX'
%%   pdf2htmlEX --zoom 1.5 --process-outline 0 resume.pdf
%%
%%%%%%%%%%%%%%%%%%%%%%%%%%%%%%%%%%%%%%%%%%%%%%%%%%%%%%%%%%%%%%%%%

% Note: Search "SIZEMARKER" to find settings you can tweak to
%       make the page, the margins, or the text bigger or smaller.


% SIZEMARKER: You may want to replace "letterpaper" with "a4paper", but when
% exporting to html, letterpaper can be printed in both letter and A4 sheets
% of paper.
% SIZEMARKER: You may also want to tweak the font size.
\documentclass[letterpaper,10pt,oneside]{article}

% Long table for page layout.
\usepackage{longtable}
\usepackage{calc}

% SIZEMARKER: Page margins
\usepackage[
left=0.37in,
right=0.65in,
top=0.85in,
bottom=0.45in,
nohead,
includefoot]{geometry}

% Hyphenation: Disabled.
\usepackage[none]{hyphenat}

% XeLaTeX packages.
%\usepackage{fontspec}
%\defaultfontfeatures{Ligatures=TeX}
%\usepackage{xunicode}
%\usepackage{xltxtra}

\renewcommand*{\familydefault}{\sfdefault}


% PDF settings and properties.

% Headers and footers: Blank header, page number in footer.
% Commented: I don't want the page number in footer.
\makeatletter
\def\ps@plain{%
\def\@oddhead{}%
\def\@evenhead{}%
\def\@oddfoot{}%
\def\@evenfoot{}}
%\def\@oddfoot{\footnotesize\hfill{Page}~{\thepage}~of~\pageref{LastPage}\hfill}%
%\def\@evenfoot{\footnotesize\hfill{Page}~{\thepage}~of~\pageref{LastPage}\hfill}}
\makeatother

\pagestyle{plain}

% Paragraph style: No indentation.
\setlength{\parindent}{0in}

% SIZEMARKER: Line spacing
\renewcommand{\baselinestretch}{1.4}

\newcommand{\DatestampY}[1]{#1}

% Macro: body (rest of the document).
% SIZEMARKER: Left vs. right columns
\newenvironment{body}
{\par\par
\begin{longtable}{p{0.125\textwidth}p{0.84\textwidth}}}
{\par\end{longtable}\par}

% Macro: section (new section for Education, Research Experience, etc.).
% SIZEMARKER: The separation here is between sections.
\renewcommand{\section}[3]{\\[-0.7cm]\pdfbookmark[2]{#2}{#3}\\%
\raggedleft  %align to the right
{\fontsize{9.5pt}{9.5pt}\selectfont\bfseries\raggedright%
\MakeUppercase{#1}}&}

% SIZEMARKER: SmallEntryGap is a small vertical gap within a long entry
\newcommand{\SmallEntryGap}{\par\vspace{0.38em}\par} 

% Round image
\usepackage{tikz}

\newcommand{\roundpic}[4][]{
\tikz\node [circle, minimum width = #2,
path picture = {
\node [#1] at (path picture bounding box.center) {
\includegraphics[width=#3]{#4}};
}] {};}


%%%%%%%%%%%%%%%%%%%%%%%%%%%%%%%%%%%%%%%%%%%%%%%%%%%%%%%%%%%%%%%%%
%% PREAMBLE.
%%%%%%%%%%%%%%%%%%%%%%%%%%%%%%%%%%%%%%%%%%%%%%%%%%%%%%%%%%%%%%%%%


% CV Info (to be customized).
\newcommand{\CVTitle}{Antonio Gámiz Delgado}
\newcommand{\CVNote}{CV compiled on {\today}}

% PDF settings and properties.
\usepackage{hyperref}
\hypersetup{
pdftitle={\CVTitle},
pdfauthor={Antonio Gámiz Delgado},
pdfcreator={},
pdfproducer={},
pdfkeywords={},
pdfpagemode={},
bookmarks=true,
unicode=true,
bookmarksopen=true,
pdfstartview=FitH,
pdfpagelayout=OneColumn,
pdfpagemode=UseOutlines,
hidelinks,
breaklinks,
colorlinks=true,
linkcolor=blue,
citecolor=blue,
filecolor=blue,
urlcolor=blue
}

%%%%%%%%%%%%%%%%%%%%%%%%%%%%%%%%%%%%%%%%%%%%%%%%%%%%%%%%%%%%%%%%%
%% ACTUAL DOCUMENT.
%%%%%%%%%%%%%%%%%%%%%%%%%%%%%%%%%%%%%%%%%%%%%%%%%%%%%%%%%%%%%%%%%

\begin{document}


%%%%%%%%%%%%%%%
% TITLE BLOCK %
%%%%%%%%%%%%%%%

\begin{body}

\raggedleft\roundpic{1.85cm}{1.85cm}{pic2.png}
&
\vspace{-2.54cm} \par
\huge{\textbf{Antonio Gámiz Delgado}} \par
\large{\textbf{Math and Computer Science Student}} \par
\normalsize{\href{https://github.com/antoniogamiz/}{@antoniogamiz} \textemdash\ Contact via \href{https://www.linkedin.com/in/antonio-gamiz}{linkedin.com/in/antonio-gamiz} or \href{mailto:antoniogamiz10@gmail.com}{antoniogamiz10@gmail.com}}
% SIZEMARKER: Separation between header and content
\vspace{0.1cm}

%%%%%%%%%%%%%%%%
%% EXPERIENCE %%
%%%%%%%%%%%%%%%%



\section{Industry}{Industry}{PDF:Industry}

\textbf{Software Engineer Internship} at \href{https://www.linkedin.com/company/zenzorrito-tecnologias}{Zenzorrito Tecnologias} \hfill \DatestampY{2020}--\DatestampY{2021} \newline
\phantom{w} Here I learned some basic about software engineering: TDD, BDD, DDD, Clean Architecture and more.\newline
\phantom{w} I worked with the following technologies: React, Django, Heroku and git/GitHub.

\textbf{Developer Student Club Member}. DSC, University of Granada \hfill \DatestampY{2019}--\DatestampY{2020} \newline
\phantom{w} Tech community to publicize open source projects and solve local problems.

\textbf{Software Intern}. Google Summer of Code \hfill \DatestampY{May 2019}--\DatestampY{August 2019} \newline
\phantom{w}Developed a \href{https://github.com/Raku/Documentable}{tool} to generate the \href{https://docs.raku.org/}{doc page} of Raku Language.

\textbf{Math Organiser}. Mathematical Student Association \hfill \DatestampY{2016}--\DatestampY{2019} \newline
\phantom{w} Organised several \href{https://sites.google.com/view/jornadasrsmeamat/iii-jornada-rsme-amat}{mathematical events}, conferences and workshops.


\SmallEntryGap

%\textbf{Engineering Manager II}. Google, Cloud Artificial Intelligence Solutions SRE\hfill \DatestampY{2020}--now\setlength{\dimen0}{\widthof{now}}\hspace{-\dimen0}\hphantom{\DatestampY{2000}} \newline
%\phantom{w}Designed and led the development of a capacity management solution for 40+ products\newline
%\phantom{w}Published a tech talk on \textit{\href{https://youtu.be/pOo0oKNM9I8}{the complexities of capacity management for distributed services}}
%\SmallEntryGap

%\textbf{Engineering Manager}. Google, Cloud Machine Learning SRE\hfill \DatestampY{2018}--\DatestampY{2020} \newline
%\phantom{w}Grew the team from four to fifteen members, trained a manager and two technical leads\newline
%\phantom{w}Improved the efficiency of artificial intelligence accelerators by 40\% across ten products\newline
%\phantom{w}Partnered with NVIDIA to advertise \href{https://cloud.google.com/blog/products/ai-machine-learning/your-ml-workloads-cheaper-and-faster-with-the-latest-gpus}{lower-cost faster artificial intelligence accelerators}\newline
%\phantom{w}Led reliability efforts across a large developer organization

\section{Skills}{Skills}{PDF:Skills}

I have experience in building webapps based on ReactJS, HTML and CSS. I have also worked  with Typescript and Javascript. I am more intestered in backend development, where I have worked with NodeJS, MySQL, Oracle and MongoDB (and Deno, the \textit{new} NodeJS). I also have a very good understanding of Git, Github and Linux. I have made some minor projects in C++ and Python. I try to follow the principles described in Clean Code and Clean Architecture and TDD. From time to time, I like to write posts about what I do in \href{https://dev.to/antoniogamiz}{DEV.to}.

%%%%%%%%%%%%%%%
%% STUDIES %%
%%%%%%%%%%%%%%%

\section{Academia}{Academia}{PDF:Academia}

\textbf{BSc Mathematics.} Current GPA 6.7/10. University of Granada\hfill \DatestampY{2016}--\DatestampY{now}\newline
\textbf{BSc Computer Science.} Current GPA 6.7/10. University of Granada\hfill \DatestampY{2016}--\DatestampY{now}

%%%%%%%%%%%%%%%
%% INTERESTS %%
%%%%%%%%%%%%%%%

\section{Interests}{Interests}{PDF:Interests}
Speaker at \href{https://github.com/antoniogamiz/discovering-perl6-talk}{esLibre} and \href{https://interferencias.tech/2019/04/20/jasyp-2019/}{Jasyp2019} \newline
Blogger at \href{https://dev.to/antoniogamiz}{DEV.to}.

\end{body}

\end{document}