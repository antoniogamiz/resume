%!TEX TS-program = xelatex
\documentclass[]{friggeri-cv}
\usepackage{afterpage}
\usepackage{hyperref}
\usepackage{color}
\usepackage{xcolor}
\usepackage{smartdiagram}
\usepackage{fontspec}
% if you want to add fontawesome package
% you need to compile the tex file with LuaLaTeX
% References:
%   http://texdoc.net/texmf-dist/doc/latex/fontawesome/fontawesome.pdf
%   https://www.ctan.org/tex-archive/fonts/fontawesome?lang=en
%\usepackage{fontawesome}
\usepackage{metalogo}
\usepackage{dtklogos}
\usepackage[utf8]{inputenc}
\usepackage{tikz}
\usetikzlibrary{mindmap,shadows}
\hypersetup{
    pdftitle={},
    pdfauthor={},
    pdfsubject={},
    pdfkeywords={},
    colorlinks=false,           % no lik border color
    allbordercolors=white       % white border color for all
}
\smartdiagramset{
    bubble center node font = \footnotesize,
    bubble node font = \footnotesize,
    % specifies the minimum size of the bubble center node
    bubble center node size = 0.5cm,
    %  specifies the minimum size of the bubbles
    bubble node size = 0.5cm,
    % specifies which is the distance among the bubble center node and the other bubbles
    distance center/other bubbles = 0.3cm,
    % sets the distance from the text to the border of the bubble center node
    distance text center bubble = 0.5cm,
    % set center bubble color
    bubble center node color = pblue,
    % define the list of colors usable in the diagram
    set color list = {lightgray, materialcyan, orange, green, materialorange, materialteal, materialamber, materialindigo, materialgreen, materiallime},
    % sets the opacity at which the bubbles are shown
    bubble fill opacity = 0.6,
    % sets the opacity at which the bubble text is shown
    bubble text opacity = 0.5,
}

\addbibresource{bibliography.bib}
\RequirePackage{xcolor}
\definecolor{pblue}{HTML}{0395DE}

\begin{document}
\header{Antonio}{Gámiz}
      {Computer Engineer}
      
% Fake text to add separator      
\fcolorbox{white}{gray}{\parbox{\dimexpr\textwidth-2\fboxsep-2\fboxrule}{%
.....
}}

% In the aside, each new line forces a line break
\begin{aside}
  \includegraphics[scale=0.18]{img/cara.png}
  \section{Contact}
    +34 639 250 244
    antoniogamiz10@gmail.com
    ~
  \section{LinkedIn \& Git}
    \href{www.linkedin.com/in/antonio-gamiz}{linkedin.com/in/antonio-gamiz}
    \href{https://github.com/antoniogamiz}{github.com/antoniogamiz}
    ~
    ~
    ~
  % use  \hspace{} or \vspace{} to change bubble size, if needed
  \section{Programming}
    \smartdiagram[bubble diagram]{
        \textbf{JS - NodeJS},
        \textbf{C/C++},
        \textbf{Python},
        \textbf{Bash},
        \textbf{Java}
    }
	~
  \section{OS Preference}
    \textbf{GNU/Linux}\includegraphics[scale=0.40]{img/5stars.png}
    \textbf{Unix}\includegraphics[scale=0.40]{img/4stars.png}
    \textbf{MacOS}\includegraphics[scale=0.40]{img/3stars.png}
    \textbf{Windows}\includegraphics[scale=0.40]{img/3stars.png}
    ~
  \section{Languages}
    \textbf{Spanish}\includegraphics[scale=0.40]{img/5stars.png}
    \textbf{English}\includegraphics[scale=0.40]{img/4stars.png}
    ~
\end{aside}
~
\section{Experience}
\begin{entrylist}
  \entry
    {10/19 - Now}
    {Developer Student Club Member}
    {University of Granada}
    {Member of the Developer Student Club of my local university. \href{https://github.com/Developer-Student-Clubs-UGR}{(more information here)}.\\}
  \entry
    {05/19 - 09/19}
    {Google of Summer of Code Program}
    {Google}
    {I successfully completed Google Summer of Code 2019, a program which connects students with mentors for 3 months to develop open source software. I did it with The Perl Foundation, \href{https://summerofcode.withgoogle.com/projects/4930522189398016}{(more information here)}.\\}
  \entry    
    {2016 - 2019}
    {Organizer in Mathematical Student Association (MSA)}
    {University of Granada}
    {Local association where we made workshops about geometry, \LaTeX, analysis, etc.\\}
\end{entrylist}
\section{Education}
~
\begin{entrylist}
  \entry
    {2016 - Now}
    {Bachelor's Degree in Computer Engineering}
    {University of Granada}
    {Currently in my forth year with a grade point average of $6,458$.\\}
  \entry    
    {2016 - Now}
    {Bachelor's Degree in Mathematics}
    {University of Granada}
    {Currently in my forth year with a grade point average of $6,458$.\\}
  \entry    
    {2014 - 2016}
    {B. Sc at IES Américo Castro, Huetor Tájar (Granada)}
    {University of Granada}
    {Successfully completed with a grade point average of $9,8$.\\}
\end{entrylist}
\section{Other Info}
~
I have a clear, logical mind with a practical approach to problem-solving and a drive to see things through to completion. I am eager to learn and contribute to Open Source community. I  really enjoy graphic design to visualize data and statistics.

From time to time I also like to give some talks or conferences such as: \href{https://github.com/antoniogamiz/Curso-Python}{Python Course}, \href{https://github.com/antoniogamiz/discovering-perl6-talk}{Discovering Perl6} in \href{https://eslib.re/2019/}{esLibre} and \href{https://github.com/antoniogamiz/oauth-talk-jasyp-2019}{OAuth2.0} in \href{https://interferencias.tech/jasyp/}{JASYP19}.

\begin{flushleft}
\emph{October 18th, 2019}
\end{flushleft}
\begin{flushright}
\emph{Antonio Gámiz}
\end{flushright}

\end{document}
